\documentclass[iop,apj]{emulateapj}
\usepackage{times}
\usepackage{graphicx}
%\usepackage{fixltx2e}
\usepackage{astrobib_mnras2e}
\usepackage{lineno}

%% preprint2 produces a double-column, single-spaced document:

%% \documentclass[preprint2]{aastex}

%% Sometimes a paper's abstract is too long to fit on the
%% title page in preprint2 mode. When that is the case,
%% use the longabstract style option.

%% \documentclass[preprint2,longabstract]{aastex}

%% If you want to create your own macros, you can do so
%% using \newcommand. Your macros should appear before
%% the \begin{document} command.
%%
%% If you are submitting to a journal that translates manuscripts
%% into SGML, you need to follow certain guidelines when preparing
%% your macros. See the AASTeX v5.x Author Guide
%% for information.

\newcommand{\cpp}{c_{\perp}}
\newcommand{\cll}{c_{||}}
\newcommand{\dpp}{d_{\perp}}
\newcommand{\gmod}{g_{\rm mod}}
\newcommand{\rmod}{r_{\rm mod}}
\newcommand{\imod}{i_{\rm mod}}
\newcommand{\gcmod}{g_{\rm cmod}}
\newcommand{\rcmod}{r_{\rm cmod}}
\newcommand{\icmod}{i_{\rm cmod}}
\newcommand{\ipsf}{i_{\rm psf}}
\newcommand{\zpsf}{z_{\rm psf}}
\newcommand{\zmod}{z_{\rm mod}}
\newcommand{\rpsf}{r_{\rm psf}}
\newcommand{\ifib}{i_{\rm fib2}}

%% You can insert a short comment on the title page using the command below.

%\slugcomment{Not to appear in Nonlearned J., 45.}

%% If you wish, you may supply running head information, although
%% this information may be modified by the editorial offices.
%% The left head contains a list of authors,
%% usually a maximum of three (otherwise use et al.).  The right
%% head is a modified title of up to roughly 44 characters.
%% Running heads will not print in the manuscript style.

\shorttitle{Galaxy Target Selection in BOSS}
\shortauthors{Padmanabhan et al.}

%% This is the end of the preamble.  Indicate the beginning of the
%% paper itself with \begin{document}.

\begin{document}
\topmargin-1cm
\linenumbers

%% LaTeX will automatically break titles if they run longer than
%% one line. However, you may use \\ to force a line break if
%% you desire.

\title{Galaxy Target Selection for the SDSS-III Baryon Oscillation Spectroscopic
        Survey}

%% author and affiliation information.
%% Note that \email has replaced the old \authoremail command
%% from AASTeX v4.0. You can use \email to mark an email address
%% anywhere in the paper, not just in the front matter.
%% As in the title, use \\ to force line breaks.

\author{Nikhil Padmanabhan}
\affil{Dept. of Physics, Yale University, 260 Whitney Ave, New Haven, CT 06511}

%% Notice that each of these authors has alternate affiliations, which
%% are identified by the \altaffilmark after each name.  Specify alternate
%% affiliation information with \altaffiltext, with one command per each
%% affiliation.

%\altaffiltext{1}{Visiting Astronomer, Cerro Tololo Inter-American Observatory.
%CTIO is operated by AURA, Inc.\ under contract to the National Science
%Foundation.}

%% Mark off your abstract in the ``abstract'' environment. In the manuscript
%% style, abstract will output a Received/Accepted line after the
%% title and affiliation information. No date will appear since the author
%% does not have this information. The dates will be filled in by the
%% editorial office after submission.

\begin{abstract}


\end{abstract}

%% Keywords should appear after the \end{abstract} command. The uncommented
%% example has been keyed in ApJ style. See the instructions to authors
%% for the journal to which you are submitting your paper to determine
%% what keyword punctuation is appropriate.

%\keywords{globular clusters: general --- globular clusters: individual(NGC 6397,
%NGC 6624, NGC 7078, Terzan 8}

\section{Introduction}

The Baryon Oscillation Spectroscopic Survey (hereafter BOSS, CITE???) is a
wide-field redshift survey designed to measure the expansion rate of the
Universe using the Baryon Acoustic Oscillation (BAO) method. On completion, BOSS
will have measured redshifts to 1.5 million galaxies and 160,000 quasars over
10,000 deg$^2$. This will allow a 1\% distance measurement using the BAO method
at z=0.35 and z=0.6 from the galaxy sample, as well as a 1.5\% distance
measurement at z=2.5 using the quasar sample. BOSS started taking spectroscopic 
data in September 2009, and is scheduled to complete in 2014. The first
spectroscopic data release from BOSS (DR9) is scheduled for July 2012; the first
cosmological results (CITES here????) were submitted in April 2012.
This paper sets out to document the algorithms used to select the BOSS galaxy
sample and describe its use as well as to describe the basic properties of the
galaxy sample.

There are three properties that determine the performance of a galaxy sample for
BAO measurements - the number density of the sample, the volume it covers, and
its large-scale clustering amplitude. These are usefully combined into the
effective volume of the sample (??? EFFECTIVE VOLUME EQN ????). Increasing the
effective volume of the sample directly increases the cosmological constraining
power of the sample. The requirements on the BOSS galaxy sample were to obtain
a sample of objects with a constant number density of $3 \times 10^{-4} h^{3}
\rm{Mpc}^{-3}$ out to z=0.6 (CHECK THIS EXACT CUT OFF????), falling off with a
magnitude limit at higher redshifts. 

Using the most luminous and most massive galaxies is an
efficient way to select samples optimized to be cosmological tracers. These
galaxies can be selected with low contamination rates from multiband imaging
data, and their redshifts may be obtained relatively efficiently. This is 
not a new observation; the Sloan Digital Sky Survey (SDSS) Luminous Red Galaxy
(LRG) sample (CITE???) and the 2SLAQ (2dF-SDSS LRG and Quasar, CITES???) used 
exactly such criteria to select samples of objects. BOSS extends these samples
in one important direction : while the SDSS LRG and 2SLAQ samples explicitly
targeted red galaxies, BOSS attempts to select a complete sample of massive
galaxies {\it irrespective} of their color. 

This paper is organized as follows FINISH THIS????

\section{The Sample Selection}

\subsection{Imaging Data}

USUAL SDSS IMAGING BOILERPLATE WOULD GO HERE.

DESCRIBE THE VARIOUS MAGNITUDES USED IN TARGET SELECTION - PSF, MODEL, FIBER2, 
AND CMODEL AND WHERE THEY GET USED.

DOCUMENT THE PARENT SAMPLE FOR TARGET SELECTION - PRIMARY OBJECTS, NO CUT ON 
PHOTOMETRICITY, WHAT FLAGS GET USED, GALAXIES.

HIGHLIGHT CHANGE IN RESOLVE, AND HOW DO YOU FIX THIS. 



\subsection{Color Definitions}

In order to succintly describe the galaxy target selection, it is convenient to
define a set of auxiliary colors that track the locus of a passively evolving
population of galaxies in $gri$ color space. Following CITE Eisenstein et al???
and Cannon et al???? CITE, we define 
\begin{eqnarray}
\cll & = &  0.7 (\gmod - \rmod) + 1.2(\rmod - \imod - 0.18)  \\
\cpp & = & (\rmod - \imod) - (\gmod - \rmod)/4.0 - 0.18 
\end{eqnarray}
to describe the low redshift locus and 
\begin{eqnarray}
\dpp & = & (\rmod - \imod) - (\gmod - \rmod)/8 \,,
\end{eqnarray}
to describe the high redshift locus. As discussed above, the colors are defined
using SDSS model magnitudes, and are extinction-corrected. As was discussed in
Eis et al CITE??, the two sets of colors are necessary to describe the color
locus first when the 4000\AA\ break is in the SDSS $g$ band and then when it
redshifts into the $r$ band at $z\sim0.4$. This naturally divides target
selection into low and high redshift samples divided approximately at
$z\sim0.4$. 

Figure ???? plots these colors compared with both the stellar locus,
as well as a population of passively evolving galaxies. We see that $\cll$ runs
parallel to the low redshift galaxy locus, while $\cpp$ is approximately
perpendicular to it. The high redshift locus is described by $\dpp$ which runs
parallel to it. 

\begin{figure}
\caption{Figure showing the various colors compared to galaxy and stellar loci.}
\label{fig:color}
\end{figure}

\subsection{The LOWZ sample}

The low redshift (denoted LOWZ) sample is designed to be a straightforward
extension of the SDSS-I/II Cut I Luminous Red Galaxy Sample (CITE???).
WRITE DOWN PRINCIPAL CUTS AND MOTIVATION BEHIND THEM - ISOLATING THE GALAXY
LOCUS AND THEN A COLOR MAGNITUDE CUT TO GET AN SELECT THE BRIGHTEST OBJECTS. 
CUT TUNED TO GET AN APPROXIMATELY VOLUME LIMITED SAMPLE. ALSO DESCRIBE MAGNITUDE
LIMITS OF THE SAMPLE.

STAR GALAXY SEPARATION.

DESCRIBE OVERLAP WITH SDSS-I/II SAMPLE. HOW BIG IS THIS OVERLAP? WE DO NOT
TARGET GALAXIES ALREADY OBSERVED IN SDSS-I/II.

DESCRIBE TS BUG. GIVE SIMPLE PRESCRIPTION FOR SKIPPING IT. SEE APPENDIX FOR
DETAILS.

\subsection{The CMASS sample}

EXPLAIN BASIC MOTIVATION BEHIND THE CMASS SAMPLE -- EXTEND CUT II LRGS AND 2SLAQ
LRGS. REMOVE COLOR CUT

EXPLAIN DPERP CUT AND SLIDING COLOR MAGNITUDE CUT

\begin{figure}
\caption{Figure showing imag vs dperp and all the various cuts we make, marked.}
\end{figure}

EXPLAIN STAR GALAXY SEPARATION -- ORIGINAL AND ADDED Z-BAND CUT.

\begin{figure}
\caption{Figure motivating the z-band star galaxy separation}
\end{figure}

EXPLAIN IFIBER2 CUT.

\subsection{Sparse Sampling Cuts}

LIST THE CUTS HERE AND MOTIVATION

\subsection{Summary of Target Selection}

We summarize the two target selection algorithms here. The LOWZ targets are
defined by 
\begin{eqnarray}
\rcmod  & < & 13.5 + \cll/0.3 \\ 
|\cpp| & < & 0.2 \\
16 & < \rcmod < & 19.6 \\
\rpsf - \rcmod & > & 0.3 \,\,,
\end{eqnarray}
while the CMASS objects are selected by 
\begin{eqnarray}
\icmod & < & 19.86 + 1.6(\dpp - 0.8) \\
17.5 & < \icmod <  & 19.9 \\
\dpp & > & 0.55 \\
\ipsf - \imod & > & 0.2 + 0.2(20 - \imod) \\
\zpsf - \zmod & > & 9.125 - 0.46 \zmod \\
\rmod - \imod & < & 2 \\
\ifib & < & 21.5 \,\,.
\end{eqnarray}

DESCRIBE CMASS SPARSE SAMPLE CUT AS WELL.

REPEAT ALL THE FLAG CUTS HERE.

\subsection{Sample Usage}

EXPLAIN BOSSTARGET1 AND ALL THE BITS

TRACK DOWN WHEN ONE OF THE BOSSTARGET1 BITS GOT OVERLOADED.

\section{Sample Properties}

\begin{figure}
\caption{The number densities of the SDSS-I/II, LOWZ and CMASS objects}
\end{figure}

\begin{figure}
\caption{The absolute magnitudes of these galaxies, compared with SDSS-I/II
objects.}
\end{figure}

DISCUSS NUMBER DENSITIES AND ABSOLUTE MAGNITUDES

DISCUSS NORTH V SOUTH (OR JUST REFERENCE PAPERS ON THIS).

\section{Conclusions}


\appendix

\section{Commissioning Data}

EXPLAIN THE ORIGINAL COMMISSIONING CUTS (ONLY FOR DATA WE PLAN ON RELEASING). 


\section{Changes in Galaxy ``RESOLVE''}

DESCRIBE THE CHANGES HERE. DOCUMENT WHEN THE CHANGE HAPPENED, AND HOW THIS 
GETS SOLVED IN THE DATA RELEASE.

\begin{figure}
\caption{An Aitoff projection showing the regions of sky where the primary
objects changed. MAKE THIS FIGURE.}
\end{figure}

\section{A Bug in the LOWZ Galaxy Selection}

DESCRIBE THE BUG IN DETAIL. WHAT CUT WAS EXACTLY MADE? WHAT DATA DID IT EFFECT?
SHOW THE EFFECT ON THE NUMBER DENSITY OF TARGETS. REPEAT HOW ONE CAN AVOID THIS.


\end{document}

