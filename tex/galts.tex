\documentclass[iop,apj]{emulateapj}


%% preprint2 produces a double-column, single-spaced document:

%% \documentclass[preprint2]{aastex}

%% Sometimes a paper's abstract is too long to fit on the
%% title page in preprint2 mode. When that is the case,
%% use the longabstract style option.

%% \documentclass[preprint2,longabstract]{aastex}

%% If you want to create your own macros, you can do so
%% using \newcommand. Your macros should appear before
%% the \begin{document} command.
%%
%% If you are submitting to a journal that translates manuscripts
%% into SGML, you need to follow certain guidelines when preparing
%% your macros. See the AASTeX v5.x Author Guide
%% for information.

%\newcommand{\vdag}{(v)^\dagger}
%\newcommand{\myemail}{skywalker@galaxy.far.far.away}

%% You can insert a short comment on the title page using the command below.

%\slugcomment{Not to appear in Nonlearned J., 45.}

%% If you wish, you may supply running head information, although
%% this information may be modified by the editorial offices.
%% The left head contains a list of authors,
%% usually a maximum of three (otherwise use et al.).  The right
%% head is a modified title of up to roughly 44 characters.
%% Running heads will not print in the manuscript style.

\shorttitle{Galaxy Target Selection in BOSS}
\shortauthors{Padmanabhan et al.}

%% This is the end of the preamble.  Indicate the beginning of the
%% paper itself with \begin{document}.

\begin{document}

%% LaTeX will automatically break titles if they run longer than
%% one line. However, you may use \\ to force a line break if
%% you desire.

\title{Galaxy Target Selection for the SDSS-III Baryon Oscillation Spectroscopic
        Survey}

%% author and affiliation information.
%% Note that \email has replaced the old \authoremail command
%% from AASTeX v4.0. You can use \email to mark an email address
%% anywhere in the paper, not just in the front matter.
%% As in the title, use \\ to force line breaks.

\author{Nikhil Padmanabhan}
\affil{Dept. of Physics, Yale University, 260 Whitney Ave, New Haven, CT 06511}

%% Notice that each of these authors has alternate affiliations, which
%% are identified by the \altaffilmark after each name.  Specify alternate
%% affiliation information with \altaffiltext, with one command per each
%% affiliation.

%\altaffiltext{1}{Visiting Astronomer, Cerro Tololo Inter-American Observatory.
%CTIO is operated by AURA, Inc.\ under contract to the National Science
%Foundation.}

%% Mark off your abstract in the ``abstract'' environment. In the manuscript
%% style, abstract will output a Received/Accepted line after the
%% title and affiliation information. No date will appear since the author
%% does not have this information. The dates will be filled in by the
%% editorial office after submission.

\begin{abstract}


\end{abstract}

%% Keywords should appear after the \end{abstract} command. The uncommented
%% example has been keyed in ApJ style. See the instructions to authors
%% for the journal to which you are submitting your paper to determine
%% what keyword punctuation is appropriate.

%\keywords{globular clusters: general --- globular clusters: individual(NGC 6397,
%NGC 6624, NGC 7078, Terzan 8}

\section{Introduction}

The Baryon Oscillation Spectroscopic Survey (hereafter BOSS, CITE???) is a
wide-field redshift survey designed to measure the expansion rate of the
Universe using the Baryon Acoustic Oscillation (BAO) method. On completion, BOSS
will have measured redshifts to 1.5 million galaxies and 160,000 quasars over
10,000 deg$^2$. This will allow a 1\% distance measurement using the BAO method
at z=0.35 and z=0.6 from the galaxy sample, as well as a 1.5\% distance
measurement at z=2.5 using the quasar sample. BOSS started taking spectroscopic 
data in September 2009, and is scheduled to complete in 2014. The first
spectroscopic data release from BOSS (DR9) is scheduled for July 2012; the first
cosmological results (CITES here????) were submitted in April 2012.
This paper sets out to document the algorithms used to select the BOSS galaxy
sample and describe its use as well as to describe the basic properties of the
galaxy sample.

There are three properties that determine the performance of a galaxy sample for
BAO measurements - the number density of the sample, the volume it covers, and
its large-scale clustering amplitude. These are usefully combined into the
effective volume of the sample (??? EFFECTIVE VOLUME EQN ????). Increasing the
effective volume of the sample directly increases the cosmological constraining
power of the sample. The requirements on the BOSS galaxy sample were to obtain
a sample of objects with a constant number density of $3 \times 10^{-4} h^{3}
\rm{Mpc}^{-3}$ out to z=0.6 (CHECK THIS EXACT CUT OFF????), falling off with a
magnitude limit at higher redshifts. 

Using the most luminous and most massive galaxies is an
efficient way to select samples optimized to be cosmological tracers. These
galaxies can be selected with low contamination rates from multiband imaging
data, and their redshifts may be obtained relatively efficiently. This is 
not a new observation; the Sloan Digital Sky Survey (SDSS) Luminous Red Galaxy
(LRG) sample (CITE???) and the 2SLAQ (2dF-SDSS LRG and Quasar, CITES???) used 
exactly such criteria to select samples of objects. BOSS extends these samples
in one important direction : while the SDSS LRG and 2SLAQ samples explicitly
targeted red galaxies, BOSS attempts to select a complete sample of massive
galaxies {\it irrespective} of their color. 

This paper is organized as follows FINISH THIS????

\section{The Sample Selection}

\subsection{Imaging Data}



\subsection{Color Definitions}

\subsection{The LOWZ sample}


\subsection{The CMASS sample}


\subsection{Summary of Target Selection}

\subsection{Sample Usage}

\section{Sample Properties}

\subsection{Redshift Distributions}


\section{Conclusions}


\appendix

\section{Commissioning Data}


\end{document}

